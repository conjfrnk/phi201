\documentclass[12pt,letterpaper]{phi201}
\usepackage{hyperref}

\begin{document}
    
\notes{Example Lecture Notes}{Connor Frank}{1/1/1970}

The following are my lecture notes from 2/12/24. I think they provide an adequate example of how to take notes using my framework. See \href{run:d:homework.tex}{homework.tex} for an example of truth tables.

\section{OR Elimination}
Going for conclusion $(P \lor Q) \to R$:
\begin{align*}
& P \lor Q
\end{align*}
\begin{align*}
& P & Q \\
& P \to R & Q \to R \\
& R & R
\end{align*}
\begin{align*}
& (P \lor Q) \to R
\end{align*}
Think about:
\begin{itemize}
    \item Strategy of order
    \item OR elimination from two separate subproofs to get the desired conclusion
\end{itemize}

\section{Truth tables}
\begin{itemize}
    \item All the answers we've gotten could have been gotten by truth tables
\end{itemize}

\section{RAA}
\begin{itemize}
    \item Last rule
    \item Had to meditate before lecture -- not a class about speculation
\end{itemize}
\subsection{Philosophy of RAA}
\begin{itemize}
    \item Strange rule
    \item Says "if, from a set of premises, you can derive a contradiction, then something in that set of premises must be false"
    \item "Indirect proof" or "proof by contradiction"
    \item If you use logic to derive something that no sane person would believe, it means that, somewhere, there is a rotten apple that you need to get rid of
\end{itemize}

\pagebreak

\subsection{Example}

Goal: To show DeMorgan's Law $\neg(P \lor Q) \vdash \neg P$ \\

\begin{logicproof}
    \pl{1;\neg(P \lor Q);A}
    \pl{2;P;A (goal: contradiction)}
    \pl{2;P \lor Q;$\lor I$ 2}
    \pl{1,2;(P \lor Q) \land \neg (P \lor Q);$\land I$ 3,1}
    \pl{1;\neg P;RA 2,4}
\end{logicproof}

\subsection{Famous Result -- Ex Falso Quodlibet (From the False, Everything Follows)}

$P, \neg P \vdash Q$ \\

\begin{logicproof}
    \pl{1;P;A}
    \pl{2;\neg P;A}
    \pl{3;\neg Q;A}
    \pl{1,2;P \land \neg P;$\land I$ 1,2}
    \pl{1,2;\neg \neg Q;RA 3,4}
    \pl{1,2;Q;DN 5}
\end{logicproof}
\begin{itemize}
    \item Don't worry---not bad logic
    \item As long as you believe consistent things, you can't do this
\end{itemize}

\subsection{Instead of $\lor E$}

$P \lor Q, \neg P \vdash Q$ \\

\begin{logicproof}
    \pl{1;P \lor Q;A}
    \pl{2;\neg P;A}
    \pl{3;\neg Q;A}
    \pl{2,3;\neg P \land \neg Q;$\land I$ 2,3}
    \pl{1,2,3;(P \lor Q) \land (\neg P \land \neg Q);$\land I$ 1,4}
    \pl{1,2;Q;RA 3,5}
\end{logicproof}

\subsection{Show $\neg(\neg P \lor Q) \vdash \neg(P \to Q)$}

\begin{logicproof}
    \pl{1;\neg(\neg P \lor Q);A}
    \pl{2;P \to Q;A (goal: RA)}
    \pl{1;\neg \neg P \land \neg Q;DM 1}
    \pl{1;\neg \neg P;$\land E$ 3}
    \pl{1;P;DN 4}
    \pl{1;\neg Q;$\land E$ 3}
    \pl{1,2;Q;MP 2,5}
    \pl{1,2;Q \land \neg Q;$\lor I$ 6,7}
    \pl{1;\neg(P \to Q);RA 2,8}
\end{logicproof}

\pagebreak

\section{Laws}
\subsection{Law of Non-Contradiction}
$\vdash \neg(P \land \neg P)$ \\

\begin{logicproof}
    \pl{1;P \land \neg P;A}
    \pl{;\neg(P \land \neg P);RA 1,1}
\end{logicproof}

\subsection{Excluded Middle}
$\vdash P \lor \neg P$ \\

\begin{logicproof}
    \pl{1;\neg(P \lor \neg P);A (goal: RA)}
    \pl{2;P;A}
    \pl{2;P \lor \neg P;$\lor I$ 2}
    \pl{1,2;(P \lor \neg P) \land \neg(P \lor \neg P);$\land I$ 1,3}
    \pl{1;\neg P;RA 2,4}
    \pl{1;P \lor \neg P;$\lor I$ 5}
    \pl{1;(P \lor \neg P) \land \neg(P \lor \neg P);$\land I$ 1,6}
    \pl{;\neg \neg(P \lor \neg P);RA 1,7}
    \pl{;P \lor \neg P;DN 8 (for non-heretical mathematicians)}
\end{logicproof}

\end{document}
