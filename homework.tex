\documentclass[12pt,letterpaper]{phi201}
\usepackage{tikz}

\begin{document}

\hw{Example}{Connor Frank}{1/1/1970}

\mytitle{Section 1 - Proof}

\myproblem{1}{1} Ex Falso Quodlibet \\
\begin{logicproof}
    \pl{1;P;A}
    \pl{2;\neg P;A}
    \pl{3;\neg Q;A}
    \pl{1,2;P \land \neg P;$\land$ I 1,2}
    \pl{1,2;\neg\neg Q;RA 3,4}
    \pl{1,2;Q;DN 5}
\end{logicproof}

\myproblem{1}{2} Disjunctive Syllogism \\
\begin{logicproof}
    \pl{1;P \lor Q;A}
    \pl{2;\neg P;A}
    \pl{3;P;A}
    \pl{2,3;Q;EFQ 2,3}
    \pl{5;Q;A}
    \pl{1,2;Q;$\lor$ E 1,3,4,5,5}
\end{logicproof}

\mytitle{Section 2 - Truth Table}

\myproblem{2}{1} Is the following a valid argument? \\
$P \lor Q, P \vdash \neg Q$ \\
\noindent\fbox{ Invalid }
\begin{multicols}{2}
    \begin{logictable}{5}{2}{1}
        \theader{P,Q,P,\lor,Q}
        \tl{1,1,1,1,1}
        \tl{1,0,1,1,0}
        \tl{0,1,0,1,1}
        \tl{0,0,0,0,0}
    \end{logictable}
    \columnbreak

    \begin{logictable}{4}{2}{0}
        \theader{P,Q,\neg,Q}
        \tl{1,1,0,1}
        \tl{1,0,1,0}
        \tl{0,1,0,1}
        \tl{0,0,1,0}
    \end{logictable}
\end{multicols}
\vspace*{-18pt}
\noindent The lines where P is true must be the same in both tables for the argument to be valid, and they are not. \\ \\

\end{document}
